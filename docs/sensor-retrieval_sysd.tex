\documentclass[a4paper]{arrowhead}

\usepackage[yyyymmdd]{datetime}
\usepackage{etoolbox}
\usepackage[utf8]{inputenc}
\usepackage{multirow}
\usepackage{hyperref}
\usepackage{float}

\renewcommand{\dateseparator}{-}

\setlength{\parskip}{1em}

%% Special references
\newcommand{\fref}[1]{{\textcolor{ArrowheadBlue}{\hyperref[sec:functions:#1]{#1}}}}
\newcommand{\mref}[1]{{\textcolor{ArrowheadPurple}{\hyperref[sec:model:#1]{#1}}}}
\newcommand{\pdef}[1]{{\textcolor{ArrowheadGrey}{#1\label{sec:model:primitives:#1}\label{sec:model:primitives:#1s}\label{sec:model:primitives:#1es}}}}
\newcommand{\pref}[1]{{\textcolor{ArrowheadGrey}{\hyperref[sec:model:primitives:#1]{#1}}}}

\newrobustcmd\fsubsection[3]{
  \addtocounter{subsection}{1}
  \addcontentsline{toc}{subsection}{\protect\numberline{\thesubsection}function \textcolor{ArrowheadBlue}{#1}}
  \renewcommand*{\do}[1]{\rref{##1},\ }
  \subsection*{
    \thesubsection\quad
    operation
    \textcolor{ArrowheadBlue}{#1}
    (\notblank{#2}{\mref{#2}}{})
    \notblank{#3}{: \mref{#3}}{}
  }
  \label{sec:functions:#1}
}
\newrobustcmd\msubsection[2]{
  \addtocounter{subsection}{1}
  \addcontentsline{toc}{subsection}{\protect\numberline{\thesubsection}#1 \textcolor{ArrowheadPurple}{#2}}
  \subsection*{\thesubsection\quad#1 \textcolor{ArrowheadPurple}{#2}}
  \label{sec:model:#2} \label{sec:model:#2s} \label{sec:model:#2es}
}

\begin{document}

%% Arrowhead Document Properties
\ArrowheadTitle{Sensor Retrieval Application System}
\ArrowheadType{System Description}
\ArrowheadTypeShort{SysD}
\ArrowheadVersion{4.6.1}
\ArrowheadDate{\today}
\ArrowheadAuthor{Jesper Frisk}
\ArrowheadStatus{RELEASE}
\ArrowheadContact{jesfri-8@student.ltu.se}
\ArrowheadFooter{\href{www.arrowhead.eu}{www.arrowhead.eu}}
\ArrowheadSetup
%%

%% Front Page
\begin{center}
  \vspace*{1cm}
  \huge{\arrowtitle}

  \vspace*{0.2cm}
  \LARGE{\arrowtype}
  \vspace*{1cm}

  %\Large{Service ID: \textit{"\arrowid"}}
  \vspace*{\fill}

  % Front Page Image
  %\includegraphics{figures/TODO}

  \vspace*{1cm}
  \vspace*{\fill}

  % Front Page Abstract
  \begin{abstract}
    This document provides system description for the \textbf{Sensor Retrieval Application System}.
  \end{abstract}

  \vspace*{1cm}

%   \scriptsize
%   \begin{tabularx}{\textwidth}{l X}
%     \raisebox{-0.5\height}{\includegraphics[width=2cm]{figures/artemis_logo}} & {ARTEMIS Innovation Pilot Project: Arrowhead\newline
%     THEME [SP1-JTI-ARTEMIS-2012-AIPP4 SP1-JTI-ARTEMIS-2012-AIPP6]\newline
%     [Production and Energy System Automation Intelligent-Built environment and urban infrastructure for sustainable and friendly cities]}
%   \end{tabularx}
%   \vspace*{-0.2cm}
 \end{center}

\newpage
%%

%% Table of Contents
\tableofcontents
\newpage
%%

\section{Overview}
\label{sec:overview}
\color{black}
This document describes the Sensor Retrieval Application System, which exists to demonstrate a created digital twin. This system will consume a service running on a digital twin that in will return the lates sensor data that has been received from the relevant physical twin.

The rest of this document is organized as follows.
In Section \ref{sec:use}, we describe the intended usage of the system.
In Section \ref{sec:delimitations}, we describe delimitations of capabilities
of the system.
In Section \ref{sec:services}, we describe the abstract service
operations consumed by the system.
In Section \ref{sec:security}, we describe the security capabilities
of the system.

\subsection{How This System Is Meant to Be Used}
\label{sec:use}

This system provides a command line interface, that makes it possible to send a request to get sensor data to the digital twin as demonstration purposes. 

\subsection{Important Delimitations}
\label{sec:delimitations}

No delimitations.

\newpage

\section{Services consumed}
\label{sec:services}

\msubsection{service}{temperature}
The Sensor Retrieval system orchestrates for a temperature service, if found a request will be sent to the digital twin, to get the temperature data.

\newpage

\section{Security}
\label{sec:security}

The security of Eclipse Arrowhead - and therefore the security of Sensor Retrieval  - is relying on X.509 certificate trust chains. The Arrowhead trust chain consists of three level:
\begin{itemize}
    \item Master certificate: \texttt{arrowhead.eu}
    \item Cloud certificate: \texttt {my-cloud.my-company.arrowhead.eu}
    \item Client certificate: \texttt{my-client.my-cloud.my-company.arrowhead.eu}
\end{itemize}

For Arrowhead certificate profile see \url{https://github.com/eclipse-arrowhead/documentation}

\newpage

\bibliographystyle{IEEEtran}
\bibliography{bibliography}

\newpage

\section{Revision History}
\subsection{Amendments}

\noindent\begin{tabularx}{\textwidth}{| p{1cm} | p{3cm} | p{2cm} | X | p{4cm} |} \hline
\rowcolor{gray!33} No. & Date & Version & Subject of Amendments & Author \\ \hline

1 & YYYY-MM-DD & \arrowversion & & Xxx Yyy \\ \hline
\end{tabularx}

\subsection{Quality Assurance}

\noindent\begin{tabularx}{\textwidth}{| p{1cm} | p{3cm} | p{2cm} | X |} \hline
\rowcolor{gray!33} No. & Date & Version & Approved by \\ \hline

1 & YYYY-MM-DD & \arrowversion  &  \\ \hline

\end{tabularx}

\end{document}